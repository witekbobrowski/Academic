\documentclass[a4paper]{article}

\usepackage{polski}
\usepackage[utf8]{inputenc}

\title{Zagadnienia do egzaminu z Metod Statystycznych}
\author{}
\date{}
\setlength{\emergencystretch}{3em}
\begin{document}

\maketitle

\begin{enumerate}
  \item Definicja procesu stochastycznego, rodzaje procesów stochastycznych, proces Markowa, macierz przejścia procesu stochastycznego i jej granica dla nieskończonego czasu.
  \item Łańcuch Markowa, równanie Chapmana-Kołomogorowa (wyprowadzenie)
  \item Stan stacjonarny, równanie definiujące, własności, sposób znajdowania stanu stacjonarnego wprost z definicji, warunek istnienia stanu stacjonarnego dla łańcucha Markowa (omówienie metod funkcji generującej i metody wykorzystującej rozwinięcie spektralne jest nieobowiązkowe)
  \item Proces liczący Bernoulliego - definicja i własności, macierz przejścia, sposób symulacji komputerowej, suma procesów Bernoulliego
  \item Proces liczący Poissona - definicja i własności, macierz przejścia, sposób symulacji komputerowej, suma procesów Poissona
  \item Systemy kolejkowe - definicja, przykłady, etapy występujące w systemie kolejkowym, wydajność, prawo Little'a, przykład zastosowania
  \item System kolejkowy Bernoulliego z jednym serwerem - definicja, własności, diagram, macierz przyjścia (wyprowadzenie), różnica pomiędzy systemami z nieskończoną i skończoną pojemnością systemu
  \item System kolejkowy M/M/1 - definicja, własności, macierz przejścia (wyprowadzenie), stan stacjonarny (wyprowadzenie)
  \item System kolejkowy Bernoulliego z wielona serwerami - definicja, własności, macierz przejścia (wyprowadzenie)
  \item System kolejkowy M/M/k - definicja, własności, macierz przejścia (wyprowadzenie), stan stacjonarny (wyprowadzenie)
  \item System kolejkowy M/M/nieskończoność - definicja, własności, macierz przejścia (wyprowadzenie), stan stacjonarny (wyprowadzenie)
  \item System kolejkowy M/M/1 (PS) - definicja, własności
  \item Algorytm Page Rank - związek z procesem stochastycznym (uproszczona macierz przejścia i propozycja Brin'a-Page'a), sposób obliczania stanu stacjonarnego dla rzeczywiscrej sieci Internetu
  \item Ukryte modele Markowa - definicja, funkcja największej wiarygodności, foreward recursion algorithm
  \item Estymacja sekwencji stanów nieobserwowalnych w ukrytych modelach Markowa - algorytm Viterbiego
\end{enumerate}

\end{document}
