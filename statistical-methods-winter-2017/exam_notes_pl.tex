\documentclass[a4paper]{article}

\usepackage{polski}
\usepackage{fullpage}
\usepackage[utf8]{inputenc}

\title{Zagadnienia do egzaminu z Metod Statystycznych}
\author{}
\date{}
\setlength{\emergencystretch}{3em}


\begin{document}

\maketitle

\tableofcontents
\clearpage

\section{Definicja procesu stochastycznego, rodzaje procesów stochastycznych, proces Markowa, macierz przejścia procesu stochastycznego i jej granica dla nieskończonego czasu.}

\section{Łańcuch Markowa, równanie Chapmana-Kołomogorowa (wyprowadzenie)}

\section{Stan stacjonarny, równanie definiujące, własności, sposób znajdowania stanu stacjonarnego wprost z definicji, warunek istnienia stanu stacjonarnego dla łańcucha Markowa (omówienie metod funkcji generującej i metody wykorzystującej rozwinięcie spektralne jest nieobowiązkowe)}

\section{Proces liczący Bernoulliego - definicja i własności, macierz przejścia, sposób symulacji komputerowej, suma procesów Bernoulliego}

\section{Proces liczący Poissona - definicja i własności, macierz przejścia, sposób symulacji komputerowej, suma procesów Poissona}

\section{Systemy kolejkowe - definicja, przykłady, etapy występujące w systemie kolejkowym, wydajność, prawo Little'a, przykład zastosowania}

\section{System kolejkowy Bernoulliego z jednym serwerem - definicja, własności, diagram, macierz przyjścia (wyprowadzenie), różnica pomiędzy systemami z nieskończoną i skończoną pojemnością systemu}

\section{System kolejkowy M/M/1 - definicja, własności, macierz przejścia (wyprowadzenie), stan stacjonarny (wyprowadzenie)}
\section{System kolejkowy Bernoulliego z wielona serwerami - definicja, własności, macierz przejścia (wyprowadzenie)}

\section{System kolejkowy M/M/k - definicja, własności, macierz przejścia (wyprowadzenie), stan stacjonarny (wyprowadzenie)}

\section{System kolejkowy M/M/nieskończoność - definicja, własności, macierz przejścia (wyprowadzenie), stan stacjonarny (wyprowadzenie)}

\section{System kolejkowy M/M/1 (PS) - definicja, własności}

\section{Algorytm Page Rank - związek z procesem stochastycznym (uproszczona macierz przejścia i propozycja Brin'a-Page'a), sposób obliczania stanu stacjonarnego dla rzeczywiscrej sieci Internetu}

\section{Ukryte modele Markowa - definicja, funkcja największej wiarygodności, foreward recursion algorithm}

\section{Estymacja sekwencji stanów nieobserwowalnych w ukrytych modelach Markowa - algorytm Viterbiego}


\end{document}
