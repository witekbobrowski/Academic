\chapter{Wstęp}

Celem pracy jest opisanie stworzonego przeze mnie projektu jakim jest biblioteka \textbf{EPUBKit}, której zadaniem jest obsługa (parsowanie oraz wyświetlanie) książek elektronicznych w formacie EPUB (Electronic Publication), a następnie wykorzystanie jej w aplikacji mobilnej na platformie iOS. Biblioteka ta została stworzona z myślą o udostępnieniu jej publicznie w formie open-source na portalu GitHub. EPUBKit jest lekkim narzędziem w języku Swift, które uprości pracę innym programistom tworzącym aplikacje na iOS rozwiązując problem jakim jest obsługa formatu EPUB. Na dzień dzisiejszy natywne biblioteki iOS nie zapewniają programistom takiego narzędzia, a publicznie istnieje niewiele rozwiązań, które cieszą się mniejszą lub większą popularnością. Problem parsowania oraz wyświetlenia publikacji w formacie EPUB nie jest trywialny. Parser musi sobie poradzić ze skomplikowaną strukturą i bardzo bogatą zawartością takiego dokumentu, który choć posiada szczegółową specyfikację, to ze względu na urozmaicenie technologii w nim wykorzystanych, czyni pracę parsera znacznie trudniejszą. EPUBKit pokonuje te trudności analizując zawartość i kolektywizując informacje, dzięki czemu finalnie tworzy instancję klasy, która reprezentuje dokument EPUB. Biblioteka dodatkowo dostarcza widok, który można wykorzystać w celu podglądu dokumentu. Cały projekt jest stworzony tak, aby jak najlepiej wpasowywał się w konwencje programowania w Swifcie na platformę iOS i opiera się na wzorcu architektonicznym Model-Widok-Kontroler, który zgodnie z wytycznymi Apple jest szeroko stosowanym przy programowaniu aplikacji na iOS. Biblioteka wykorzystuje w pełni język Swift i wszelkie jego nowinki, które sprawiają że jest językiem bardzo praktycznym a przy tym lekkim, szybkim i przyjaznym. Biblioteka EPUBKit choć nie jest idealna, a na pewno nie będąca w swojej finalnej wersji, to już w aktualnej formie oferowane przez nią API (Application Programming Interface) czyni bibliotekę funkcjonalną oraz konkurencyjną. Ze względu na wciąż ewoluującą oraz szeroką specyfikację formatu EPUB dostarczenie takiej biblioteki, która gwarantowała by taką funkcjonalność oraz była by tak rozwinięta jak, dla przykładu program iBooks na iOS wymagało by ogromnego nakładu pracy ze strony znacznego zespołu programistów. Głęboko wierzę w słuszność ruchu wolnego oprogramowania, dlatego też postępuję zgodnie z jego duchem i decyduję się upublicznić moją pracę, która już w obecnej formie może zaoszczędzić wiele czasu komuś, kto natknie się na problem obsługi dokumentu EPUB, a przy tym może zostać przez tą osobę ulepszona i rozwinięta.

Mobilny system od Apple, iOS niedługo będzie obchodził 10 lat od wprowadzenia go na rynek. W 2015 roku Apple ogłosiło, że dostarczyli konsumentom już ponad miliard urządzeń z tym właśnie systemem a dziś jest ich z pewnością znacznie więcej\cite{bilion-iphone}. Każdy kolejny model telefonu komórkowego marki Apple, prezentowany z roczną częstotliwością cieszy się coraz większym powodzeniem. Oprócz nowych urządzeń dostajemy w pakiecie nową wersję systemu iOS, która jest nie tylko udoskonaleniem poprzedniej wersji, ale również jej pełnoprawnym następcą wprowadzając nowy zbiór zarówno funkcjonalności jak i elementów wizualnych. Tak dynamicznie rozwijający się system jest bardzo atrakcyjny dla użytkownika, który dzięki darmowym aktualizacjom dla starszych urządzeń wciąż może cieszyć się najnowszym oprogramowaniem. iOS Development staje się coraz popularniejszy wśród programistów oraz przyciąga wielu młodych ludzi zainteresowanych technologią. Język Swift, nad którym prace zostały rozpoczęte w 2010 roku przez Chrisa Lattnera oraz wielu innych programistów z Apple, a w 2014 roku miał swój debiut, dzisiaj jest już głównym językiem programowania mobilnych aplikacji na platformę iOS oraz aplikacji \textit{desktopowych} na macOS i wyparł dotychczas używany w tych celach język Objective-C, który swoją historię ma również ściśle związaną z Apple. Korzenie fundamentalnych frameworków z iOS, takich jak CocoaTouch, sięgają lat 80-tych poprzedniego stulecia, a przez ten czas były bardzo silnie rozwijane i wykorzystywane w systemie macOS. Nowoczesny język oraz potężne SDK (Software Development Kit) stanowią dziś podstawę pracy z aplikacjami na te platformy. Z roku na rok, wraz z nową wersją systemu, Apple uaktualnia istniejące API oraz dostarcza nowe biblioteki zapewniające dostęp do najnowszych elementów systemu.

Ta praca dokumentuje bibliotekę \textbf{EPUBKit}, rozpoczynając od dokładnego opisu środowiska, wykorzystanych narzędzi, scharakteryzowano format EPUB i jego specyfikację techniczną, opisano proces tworzenia biblioteki, jej strukturę oraz możliwości dystrybucji biblioteki jako moduł gotowy do wykorzystania przez developerów. Następnie w celu demonstracji funkcjonalności biblioteki opisano proces tworzenia aplikacji z jej wykorzystaniem.
