\chapter{Charakterystyka EPUB}

\section{Omówienie}

EPUB jest standardem formatu dystrybucji cyfrowych publikacji i dokumentów opartych na standardach technologii webowej. EPUB definiuje
formę reprezentacji, organizacji struktury oraz kodowania określonej zawartości webowej, na co składają się XHTML, CSS, SVG, obrazy i
inne zasoby sprowadzone do formy pojedyńczego pliku. EPUB daje wydawcom możliwość stworzenia cyfrowej publikacji a następnie dysrybuowania
go, a odbiorcy łatwy dostęp do pliku niezależnie od urządzenia jakim operuje. Jako następca OEB (Open eBook Publication Structure),
zaprezentowanego w 1999 roku, EPUB 2 został ustandaryzowany w roku 2007 a aktualną jego wersją jest EPUB 3.1 (styczeń 2017). Dzisiaj jest
on standardem wykorzystywanym na szeroką skale przez wszystkich wydawców. Obok MOBI oraz PDF dominuje rynek, dzięki jego popularności
wsród wydawców oraz wsparciu urządzeń. W przeciwieństwie do MOBI które zostało spopularyzowane przez Amazon, właściciela sklepu amazon.com,
giganta dystybucji książek elektronicznych oraz producenta czytników elentronicznych marki Kindle, które dominują swój rynek, EPUB jest
standardem uniwersalnym, nieograniczonym do jednej platformy. EPUB zarówna jak i MOBI charakteryzuje się tym, że jego zawartość nie jest
statyczna, co oznacza, że ilość która jest wyświetlana dopasowana jest do wielkośći ekranu urządzenia dzięki czemu jest ona bardziej
przyjazna dla odbiorcy (PDF natomiast jest już podzielony na strony których nie da się podzielić). Popularną praktyką wśród dystrybutorów
książek elektronicznych, jest dostarczanie ksiązki klientowi który ją zakupił we wszystkich trzech wcześniej wymienionych formatach.
EPUB jako standard jest szeroko udokumentowany dzięki takim stronom jak \href{idpf.org}{International Digital Publishing Forum} czy
\href{epubzone.org}{EPUBZone}. W następnej sekcji zostanie szczegółowo opisana struktura formatu .epub.

\section{Specyfikacja}
