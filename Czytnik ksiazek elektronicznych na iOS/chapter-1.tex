\chapter{Wstęp}

\chapter{Wstęp}

Mobilny system od Apple, iOS niedługo będzie obchodził 10 lat od wprowadzenia go na rynek. W 2014 roku Apple ogłosiło, że jest ponad miliard aktywnych urządzeń z tym właśnie systemem a dziś jest ich z pewnością znacznie więcej. Każdy kolejny model telefonu komórkowego marki Apple, prezentowany z roczną częstotliwością cieszy się coraz większym powodzeniem. Oprócz nowych urządzeń dostajemy w pakiecie nową wersję systemu iOS która jest nie tylko udoskonaleniem poprzedniej wersji, ale również jej pełnoprawnym następcą wprowadzając nowy zbiór zarówno funkcjonalności jak i elementów wizualnych. Tak dynamicznie rozwijający się system jest bardzo atrakcyjny dla użytkownika, który dzięki darmowym aktualizacjom dla starszych urządzeń wciąż może cieszyć się najnowszym oprogramowaniem. iOS Development staje się coraz popularniejszy wśród programistów oraz przyciąga wielu młodych ludzi zainteresowanych technologią. Język Swift nad którym prace zostały rozpoczęte w 2010 roku przez Chrisa Lattnera oraz wielu innych programistów z Apple, a w 2014 roku miał swój debiut, dzisiaj jest już głównym językiem programowania mobilnych aplikacji na platformę iOS oraz aplikacji desktopowych na MacOS i wyparł dotychczas używany w tych celach język Objective-C, który swoją historię ma również ściśle związaną z Apple. Korzenie fundamentalnych frameworków z iOS, takich jak CocoaTouch, sięgają lat 80-tych poprzedniego stulecia, a przez ten czas były bardzo silnie rozwijane i wykorzystywane w desktopowym systemie MacOS. Nowoczesny język oraz potężne SDK (Software Development Kit) stanowią dziś podstawę pracy z aplikacjami na te platformy. Z roku na rok, wraz z nową wersją systemu, Apple uaktualnia istniejące API (Application Programming Interface) oraz dostarcza nowe biblioteki zapewniające dostęp do najnowszych elementów systemu.


Ta praca dokumentuje framework "EPUBKit" którego zadaniem jest obsługa (parsowanie oraz wyświetlanie) książek elektronicznych w formacie EPUB (Electronic Publication) a następnie wykorzystanie go w aplikacji. Rozpoczynając od dokładnego opisu środowiska, wykorzystanych narzędzi, scharakteryzowano format EPUB i jego spycyfikację techniczną, opisano proces tworzenia frameworku, jego strukturę oraz możliwości dystrybucji biblioteki jako moduł gotowy do wykorzystania przez developerów. Następnie w celu demonstracji funkcjonalności frameworku opinasno proces tworzenia aplikacji z jego wykorzystaniem. Celem pracy jest stworzenie prostego i lekkiego narzędzia w języku Swift, które uprości pracę innym programistom tworzącym aplikacje na iOS rozwiązująć problem jakim jest obługa formatu EPUB. Na dzień dzisiejszy natywne biblioteki iOS nie zapewniają programistom takiego narzędzia, a publicznie istnieje niewiele rozwiązań, które cieszą się mniejszą lub większą popularnością.
