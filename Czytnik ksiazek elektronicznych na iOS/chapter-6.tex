\chapter{Podsumowanie}

Celem tej pracy było opisanie stworzonego przeze mnie projektu \textbf{EPUBKit}, który jest biblioteką napisaną w swiftcie na iOS. Starałem się również w całej pracy przedstawić sam język swift, ponieważ dla wielu jest on nieznany ze względu na to, że jeszcze jest nowością. Ta praca dokumentuje również specyfikację formatu publikacji elektronicznej EPUB, oraz pokazuje w jaki sposób należy obchodzić się z tym formatem w przypadku, gdy chcemy wyciągnąć z niego pewne informacje. \textbf{EPUBKit} na jego podstawie budował zupełnie nową strukturę danych w formie klasy \texttt{EPUBDocument}, którą można następnie wyświetlić przy pomocy klas widoku biblioteki.

Pomimo tego, że EPUB jest tak mocno wyspecyfikowanym formatem, różnice między wersjami (głownie 2.0 versus 3.0) są dość znaczące co nie powinno dziwić zważając na okres czasu, które je dzielił. Największą trudność przy parsowaniu i wyświetlaniu takiego dokumentu, sprawia jednak nie różnica między specyfikacjami, lecz luźne interpretacje formatu stosowane przez twórców. W wielu aspektach publikacje różnią się od siebie co znacznie utrudniało moją pracę, szczególnie w przypadku implementacji widoku. Koniec końców musiałem uciekać się do takich środków jak chociażby umieszczanie własnych elementów \textit{HTML} pomiędzy dokumentami publikacji, w celu ujednolicenia sposobu nawigacji po dokumencie. Format EPUB jest fascynującą technologią, myślę, że szczególnie ze względu na to iż jest rozwijany przez środowisko \textit{open-source}, lecz przede wszystkim ze względu na jego możliwości. Wyprzedza on znacznie takie formaty jak MOBI, i na pewno będzie on standardem, który nadal będzie dominował na rynku publikacji elektronicznych.

Programowanie na iOS jest jednak ważniejszym tematem tej pracy, niż sam format EPUB. Przedstawiłem stan tej technologii na dzień dzisiejszy, i starałem się wytłumaczyć wszystkie jej najbardziej elementarne paradygmaty w ramach rzetelnego raportu. Platforma ta rozwija się coraz mocniej z roku na rok, i nic nie wskazuje na to iż miało by się to zmienić.
